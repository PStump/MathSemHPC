%
% standalone.tex -- 
%
% (c) 2012 Prof. Dr. Andreas Mueller, HSR
% $Id: ws-skript.tex,v 1.34 2008/11/02 22:46:16 afm Exp $
%
%\documentclass[a4paper,12pt]{book}
\documentclass[a4paper]{book}
\usepackage{geometry}
\geometry{papersize={170mm,240mm},total={130mm,200mm}}
\usepackage{ngerman}
\usepackage{times}
\usepackage{amsmath}
\usepackage{amssymb}
\usepackage{amsfonts}
\usepackage{amsthm}
\usepackage{graphicx}
\usepackage{fancyhdr}
\usepackage{textcomp}
\usepackage[all]{xy}
\usepackage{txfonts}
\usepackage{alltt}
\usepackage{verbatim}
\usepackage{paralist}
\usepackage{makeidx}
\usepackage{array}
\usepackage{hyperref}
%\usepackage{tikz}
%\usetikzlibrary{arrows,decorations.pathmorphing,positioning,fit,petri}
%\usetikzlibrary{calc,intersections,through,backgrounds,graphs}
%\usetikzlibrary{patterns,decorations.pathreplacing}
%\usetikzlibrary{shapes,snakes,trees}
%\usetikzlibrary{decorations.pathreplacing}
%\usetikzlibrary{patterns}
\usepackage{listings}
\lstdefinestyle{Matlab}{
  numbers=left,
  belowcaptionskip=1\baselineskip,
  breaklines=true,
  frame=L,
  xleftmargin=\parindent,
  language=Matlab,
  showstringspaces=false,
  basicstyle=\footnotesize\ttfamily,
  keywordstyle=\bfseries\color{green!40!black},
  commentstyle=\itshape\color{purple!40!black},
  identifierstyle=\color{blue},
  stringstyle=\color{orange},
  numberstyle=\ttfamily\tiny
}
\usepackage{caption}
\usepackage{subcaption}
%\usepackage{cite}
\usepackage{standalone}
\usepackage[sorting=none,backend=bibtex]{biblatex}
\AtEndDocument{\clearpage\ifodd\value{page}\else\null\clearpage\fi}
\begin{document}
%\bibliographystyle{plain}
\pagestyle{fancy}
\hypersetup{
    colorlinks=true,
    linktoc=all,
    linkcolor=blue
}
\newcounter{beispiel}
\newenvironment{beispiele}{
\bgroup\smallskip\parindent0pt\bf Beispiele\egroup

\begin{list}{\arabic{beispiel}.}
  {\usecounter{beispiel}
  \setlength{\labelsep}{5mm}
  \setlength{\rightmargin}{0pt}
}}{\end{list}}

\newenvironment{teilaufgaben}{
\begin{enumerate}
\renewcommand{\labelenumi}{\alph{enumi})}
}{\end{enumerate}}
% Loesung
\def\swallow#1{
%nothing
}
\newenvironment{loesung}{%
\begin{proof}[L"osung]%
\renewcommand{\qedsymbol}{$\bigcirc$}
}{\end{proof}}
\def\keineloesungen{%
\renewenvironment{loesung}{\swallow\begingroup}{\endgroup}%
}

\newenvironment{beispiel}{%
\begin{proof}[Beispiel]%
\renewcommand{\qedsymbol}{$\bigcirc$}
}{\end{proof}}

\newtheorem{satz}{Satz}[chapter]
\newtheorem{hilfssatz}{Hilfssatz}[chapter]
\newtheorem{definition}{Definition}[chapter]
\newtheorem{annahme}{Annahme}[chapter]

\def\chapterauthor#1{{\large #1}\bigskip\bigskip}
% To compile a single article, create a file input.tex, which contains
% a command to input your article
\input input.tex

\end{document}
