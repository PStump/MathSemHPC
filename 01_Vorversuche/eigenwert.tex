%%% Local Variables: 
%%% mode: latex
%%% TeX-master: "00_main"
%%% End: 


\section{Eigenwert und Eigenvektor}

Was sind Eigenwert und Eigenvektor?  Folgendes Beispiel soll diese Frage beantworten.

% Beispiel aus http://de.wikibooks.org/wiki/Mathematik:_Lineare_Algebra:_Eigenwerte_und_Eigenvektoren
\begin{beispiel}
  Gegeben seien:
  \begin{align*}
    A &= \begin{pmatrix}
           1 & 2\\
           0 & -2
         \end{pmatrix}
      &x &= \begin{pmatrix}
              -2\\
              3
            \end{pmatrix}
      &y &= \begin{pmatrix}
              1\\
              0
            \end{pmatrix} 
  \end{align*}
  Wir betrachten nun $A \cdot x$ und $A \cdot y$:
  \begin{align*}
    A \cdot x &= \begin{pmatrix}
                   1 & 2\\
                   0 & -2
                 \end{pmatrix}
                 \begin{pmatrix}
                   -2\\
                   3
                 \end{pmatrix}
               = \begin{pmatrix}
                   4\\
                   -6
                 \end{pmatrix}\\
    A \cdot y &= \begin{pmatrix}
                   1 & 2\\
                   0 & -2
                 \end{pmatrix}
                 \begin{pmatrix}
                   1\\
                   0
                 \end{pmatrix}
               = \begin{pmatrix}
                   1\\
                   0
                 \end{pmatrix}
  \end{align*}
  Interessant ist jetzt, dass:
  \begin{align*}
    A \cdot x &= -2x = (-2)\cdot\begin{pmatrix}
                                  -2\\
                                  3
                                \end{pmatrix}
                              = \begin{pmatrix}
                                  4\\
                                  -6
                                \end{pmatrix}\\
    A \cdot y &= 1y = 1\cdot\begin{pmatrix}
                              1\\
                              0
                            \end{pmatrix}
                          = \begin{pmatrix}
                              1\\
                              0
                            \end{pmatrix}
  \end{align*}
  Daraus ist ersichtlich, dass eine Matrix $A$, welche mit einem Vektor $x$ multipliziert wird, das gleiche Ergebnis erhält, wie wenn der Vektor $x$ mit einer Konstante multipliziert wird.  (Natürlich gilt das gleiche auch für $y$.)  Die Konstanten $1$ und $-2$ nennt man \begriff{Eigenwerte} der Matrix $A$.  Die Vektoren $x$ und $y$ heissen \begriff{Eigenvektoren} der Matrix $A$.
\end{beispiel}