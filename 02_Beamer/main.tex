%%%%%%%%%%%%%%%%%%%%%%%%%%%%%%%%%%%%%%%%%%%%%%%%%%%%%%%%%%%%%%%%%%%%%%
%         FILE:  main.tex
%
%        USAGE:  ---
%
%  DESCRIPTION:  This file is used for testing the beamer class.
%
%      OPTIONS:  see different input's below
% REQUIREMENTS:  full LaTeX installations
%         BUGS:  ---
%        NOTES:  ---
%       AUTHOR:  Pascal Stump, pascal-stump@bluewin.ch
%      VERSION:  see git repo
%      CREATED:  23.Apr.2014 - 22:06
%     REVISION:  see git repo
%%%%%%%%%%%%%%%%%%%%%%%%%%%%%%%%%%%%%%%%%%%%%%%%%%%%%%%%%%%%%%%%%%%%%%

%%% LaTeX PREAMBLE
\documentclass[compress,
               red]
              {beamer}
\mode<presentation>
{
  \usetheme{Frankfurt}
}

\input{headers/normalHeader}

%%% other packages
\usepackage{standalone}

\setcounter{MaxMatrixCols}{20}
\definecolor{colKeys}{rgb}{0,0,1}
\definecolor{colIdentifier}{rgb}{0,0,0}
\definecolor{colComments}{rgb}{0,1,0.3}
\definecolor{colString}{rgb}{0,0.5,0}

\definecolor{dkgreen}{rgb}{0,0.6,0}
\definecolor{gray}{rgb}{0.5,0.5,0.5}
\usepackage{listings}
\lstset{language=Matlab,
   keywords={break,case,catch,continue,else,elseif,end,for,function,
   global,if,otherwise,persistent,return,switch,try,while},
   float=hbp,
   basicstyle=\ttfamily\tiny,
   identifierstyle=\color{colIdentifier},
   keywordstyle=\color{blue},
   commentstyle=\color{dkgreen},
   stringstyle=\color{dkgreen},
   columns=flexible,
   tabsize=2,
   frame=single,
   numbers=left,
   extendedchars=true,
   showspaces=false,
   numberstyle=\tiny\color{gray},
   stepnumber=1,
   numbersep=10pt,
   showspaces=false,
   showstringspaces=false,
   breakautoindent=true}

\usetikzlibrary{positioning,
                arrows,
                shapes}
   
%%% BibTeX document
%\bibliography{testTex/testBibtex}
%%% Title
\title{MapReduce}
\subtitle{MapReduce und Eigenwert-Parallelisierung}
\author{Daniel Monti \and Pascal Stump}
\institute{HSR Hochschule für Technik Rapperswil}
\date{5.\,Mai 2014}


%%%%%%%%%%%%%%%%%%%%%%%%%%%%%%%%%%%%%%%%%%%%%%%%%%%%%%%%%%%%%%%%%%%%%%
%%% BEGIN DOCUMENT
\begin{document}

%%% input tex files
\frame{
  \titlepage
}

\frame{
  \tableofcontents
}

%\frame{
%  \frametitle{Gut zur info}
%  \url{http://www.math.utah.edu/~smith/AmberSmith_GSAC_Beamer.pdf}
%  zeigt gut was möglich ist.
%}

\section{MapReduce}
\subsection{Definition}
\frame{
  \frametitle{Was ist MapReduce}
}


%%% Local Variables: 
%%% mode: latex
%%% TeX-master: "main"
%%% End: 


\section{Eigenwert Berechnung}
%%% Local Variables: 
%%% mode: latex
%%% TeX-master: "00_main"
%%% End: 


\section{Eigenwert und Eigenvektor}

Was sind Eigenwert und Eigenvektor?  Folgendes Beispiel soll diese Frage beantworten.

% Beispiel aus http://de.wikibooks.org/wiki/Mathematik:_Lineare_Algebra:_Eigenwerte_und_Eigenvektoren
\begin{beispiel}
  Gegeben seien:
  \begin{align*}
    A &= \begin{pmatrix}
           1 & 2\\
           0 & -2
         \end{pmatrix}
      &x &= \begin{pmatrix}
              -2\\
              3
            \end{pmatrix}
      &y &= \begin{pmatrix}
              1\\
              0
            \end{pmatrix} 
  \end{align*}
  Wir betrachten nun $A \cdot x$ und $A \cdot y$:
  \begin{align*}
    A \cdot x &= \begin{pmatrix}
                   1 & 2\\
                   0 & -2
                 \end{pmatrix}
                 \begin{pmatrix}
                   -2\\
                   3
                 \end{pmatrix}
               = \begin{pmatrix}
                   4\\
                   -6
                 \end{pmatrix}\\
    A \cdot y &= \begin{pmatrix}
                   1 & 2\\
                   0 & -2
                 \end{pmatrix}
                 \begin{pmatrix}
                   1\\
                   0
                 \end{pmatrix}
               = \begin{pmatrix}
                   1\\
                   0
                 \end{pmatrix}
  \end{align*}
  Interessant ist jetzt, dass:
  \begin{align*}
    A \cdot x &= -2x = (-2)\cdot\begin{pmatrix}
                                  -2\\
                                  3
                                \end{pmatrix}
                              = \begin{pmatrix}
                                  4\\
                                  -6
                                \end{pmatrix}\\
    A \cdot y &= 1y = 1\cdot\begin{pmatrix}
                              1\\
                              0
                            \end{pmatrix}
                          = \begin{pmatrix}
                              1\\
                              0
                            \end{pmatrix}
  \end{align*}
  Daraus ist ersichtlich, dass eine Matrix $A$, welche mit einem Vektor $x$ multipliziert wird, das gleiche Ergebnis erhält, wie wenn der Vektor $x$ mit einer Konstante multipliziert wird.  (Natürlich gilt das gleiche auch für $y$.)  Die Konstanten $1$ und $-2$ nennt man \begriff{Eigenwerte} der Matrix $A$.  Die Vektoren $x$ und $y$ heissen \begriff{Eigenvektoren} der Matrix $A$.
\end{beispiel}

\section{Schluss}
\frame{
  \frametitle{Schlussbemerkung}

  \begin{itemize}
    \item Etwas parallelisieren ist aufwendig
    \item MapReduce für Teilparallelisierung grosser Daten
  \end{itemize}
}

\subsection{Fragen}
\frame{
  \frametitle{Fragen}
}




%%% Local Variables: 
%%% mode: latex
%%% TeX-master: "main"
%%% End: 


%%% END DOCUMENT
\end{document}