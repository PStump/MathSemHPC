%%%%%%%%%%%%%%%%%%%%%%%%%%%%%%%%%%%%%%%%%%%%%%%%%%%%%%%%%%%%%%%%%%%%%%
%         FILE:  main.tex
%
%        USAGE:  ---
%
%  DESCRIPTION:  This file is used for testing the beamer class.
%
%      OPTIONS:  see different input's below
% REQUIREMENTS:  full LaTeX installations
%         BUGS:  ---
%        NOTES:  ---
%       AUTHOR:  Pascal Stump, pascal-stump@bluewin.ch
%      VERSION:  see git repo
%      CREATED:  23.Apr.2014 - 22:06
%     REVISION:  see git repo
%%%%%%%%%%%%%%%%%%%%%%%%%%%%%%%%%%%%%%%%%%%%%%%%%%%%%%%%%%%%%%%%%%%%%%

%%% LaTeX PREAMBLE
\documentclass[compress,
               red]
              {beamer}
\mode<presentation>
{
  \usetheme{Frankfurt}
}

%%%%%%%%%%%%%%%%%%%%%%%%%%%%%%%%%%%%%%%%%%%%%%%%%%%%%%%%%%%%%%%%%%%%%%
%         FILE:  normalHeader.tex
%
%        USAGE:  ---
%
%  DESCRIPTION:  normal header for LaTeX files
%
%      OPTIONS:  ---
% REQUIREMENTS:  full LaTeX installations
%                main document
%                biber for bibliography
%         BUGS:  - change ngerman babel to nswissgerman (when
%                    cygwin/texlive is updated
%                - no support of auto language detection (with the
%                    current cygwin/texlive version)
%        NOTES:  ---
%       AUTHOR:  Pascal Stump, pascal-stump@bluewin.ch
%      VERSION:  see git repo
%      CREATED:  18.Apr.2014 - 17:50
%     REVISION:  see git repo
%%%%%%%%%%%%%%%%%%%%%%%%%%%%%%%%%%%%%%%%%%%%%%%%%%%%%%%%%%%%%%%%%%%%%%

%%% language definitions
\usepackage[T1]{fontenc}         % Umlaute as one character
\usepackage[utf8]{inputenc}      % character encoding

\usepackage[main=ngerman,        % main language in document
            english,             % other languages
            ngerman]
           {babel}               % correct language support
                                 %   change language within document :
                                 %     \selectlanguage{}

\usepackage[babel,               % use babel in background
            german=swiss]        % define german's style, also:
                                 %   quotes; guillemets; swiss
           {csquotes}            % quotes standardisation for whole
                                 %   document, usage: \enquote{}
                                 %   csquotes does not now
                                 %   nswissgerman, therefor definition:
\DeclareQuoteAlias{german}{nswissgerman}
                                 %   definition below only works with
                                 %     babel=german
\defineshorthand{"`}{\openautoquote}
\defineshorthand{"'}{\closeautoquote}

%%% miscellaneous
\usepackage{hyperref}            % url, clickable pdf

\usepackage[style=ieee,
            backend=biber,
            language=auto]
            {biblatex}           % citing of references
                                 %   usage:
                                 %     \bibliography{pathTo}
                                 %     \cite[prenote][postnote]{key}
                                 %     \printbibliography
\DeclareLanguageMapping{nswissgerman}{ngerman}

%%% preview
\usepackage[showlables,sections,floats,textmath,displaymath]{preview}

%%% picture & graphics
\usepackage{tikz}                % TikZ graphics
% \usetikzlibrary{arrows,        % include into main only what needed
%                 decorations,
%                 pathmorphing,
%                 backgrounds,
%                 positioning,
%                 fit,
%                 petri,
%                 calc,
%                 intersections,
%                 through}

%%% units
\usepackage{siunitx}             % SI unit  \si{unit} or \SI{value}{unit}
\sisetup{per-mode=symbol-or-fraction,
         detect-shape,
         range-phrase = { \translate{bis} } }
                                 %   fraction; font shape auto detect

%%% chemistry
\usepackage[version=3]{mhchem}   % easy typesetting of chemical
                                 %   formula, usage:
                                 %     \ce{1/2H20}
                                 %     \ce{^{227}_{90}Th+}

\usepackage{chemfig}             % drawing molecules (with TikZ)
                                 %   usage:
                                 %     \chemfig{}
% http://mirror.switch.ch/ftp/mirror/tex/macros/generic/chemfig/chemfig_doc_en.pdf

%%% electronic                   % drawing electrical circuits (with
\usepackage[europeanvoltages,
            europeancurrents,
            europeanresistors,   % rectangular shape
            americaninductors,   % "4-bumbs" shape
            europeanports,       % rectangular logic ports
            siunitx,             % #1<#2>
            emptydiodes,
            noarrowmos,
            smartlabels]         % lables are rotated in a smart way
           {circuitikz}          %   TikZ), usage:
                                 %   \begin{circuitikz}
% http://mirror.switch.ch/ftp/mirror/tex/graphics/pgf/contrib/circuitikz/circuitikzmanual.pdf



%%% other packages
\usepackage{standalone}

\setcounter{MaxMatrixCols}{20}
\definecolor{colKeys}{rgb}{0,0,1}
\definecolor{colIdentifier}{rgb}{0,0,0}
\definecolor{colComments}{rgb}{0,1,0.3}
\definecolor{colString}{rgb}{0,0.5,0}

\definecolor{dkgreen}{rgb}{0,0.6,0}
\definecolor{gray}{rgb}{0.5,0.5,0.5}
\usepackage{listings}
\lstset{language=Matlab,
   keywords={break,case,catch,continue,else,elseif,end,for,function,
   global,if,otherwise,persistent,return,switch,try,while},
   float=hbp,
   basicstyle=\ttfamily\tiny,
   identifierstyle=\color{colIdentifier},
   keywordstyle=\color{blue},
   commentstyle=\color{dkgreen},
   stringstyle=\color{dkgreen},
   columns=flexible,
   tabsize=2,
   frame=single,
   numbers=left,
   extendedchars=true,
   showspaces=false,
   numberstyle=\tiny\color{gray},
   stepnumber=1,
   numbersep=10pt,
   showspaces=false,
   showstringspaces=false,
   breakautoindent=true}

\usetikzlibrary{positioning,
                arrows}
   
%%% BibTeX document
%\bibliography{testTex/testBibtex}
%%% Title
\title{MapReduce}
\subtitle{MapReduce und Eigenwert-Parallelisierung}
\author{Daniel Monti \and Pascal Stump}
\institute{HSR Hochschule für Technik Rapperswil}
\date{5.\,Mai 2014}


%%%%%%%%%%%%%%%%%%%%%%%%%%%%%%%%%%%%%%%%%%%%%%%%%%%%%%%%%%%%%%%%%%%%%%
%%% BEGIN DOCUMENT
\begin{document}

%%% input tex files
\frame{
  \titlepage
}

%\frame{
%  \frametitle{Gut zur info}
%  \url{http://www.math.utah.edu/~smith/AmberSmith_GSAC_Beamer.pdf}
%  zeigt gut was möglich ist.
%}

\section{MapReduce}
\subsection{Definition}
\frame{
  \frametitle{Was ist MapReduce}

  \begin{itemize}
    \item Programmiermodell
    \item Parallele Verarbeitung von Petabyte an Daten
    \item Google
    \item Cluster aus \enquote{normaler Hardware}
    \item MapReduce-Framework (z.B. hadoop) sorgt für Aufteilung
  \end{itemize}
}

\frame{
  \frametitle{Datenfulss}
  \begin{center}
  \begin{tikzpicture}
    \node[rectangle,fill=blue!20] (data) {Daten};

    \node[circle,fill=green!80] (map1) [right=of data,yshift=1.5cm,xshift=-.5cm] {Map};
    \node[circle,fill=green!80!black] (map2) [right=of data,xshift=-.5cm] {Map};
    \node[circle,fill=green!60!black] (map3) [right=of data,yshift=-1.5cm,xshift=-.5cm] {Map};

    \node[rectangle,fill=blue!20,align=left] (zData) [right=of map2,xshift=-.5cm] {Zwischen-\\ergebnisse};

    \node[ellipse,fill=orange!40] (reduce1) [right=of zData,yshift=1cm,xshift=-.5cm] {Reduce};
    \node[ellipse,fill=orange!60] (reduce2) [right=of zData,yshift=-1cm,xshift=-.5cm] {Reduce};

    \node[ellipse,fill=orange!40] (erg1) [right=of reduce1,xshift=-.5cm] {Ergebnis};
    \node[ellipse,fill=orange!60] (erg2) [right=of reduce2,xshift=-.5cm] {Ergebnis};

    \draw[->,>=stealth',thick] (data)  to [out=45,in=180]  (map1);
    \draw[->,>=stealth',thick] (data)  to [out=0,in=180]   (map2);
    \draw[->,>=stealth',thick] (data)  to [out=-45,in=180] (map3);

    \draw[->,>=stealth',thick] (map1)  to [out=0,in=135]   (zData);
    \draw[->,>=stealth',thick] (map2)  to [out=0,in=180]   (zData);
    \draw[->,>=stealth',thick] (map3)  to [out=0,in=-135]  (zData);

    \draw[->,>=stealth',thick] (zData) to [out=45,in=180]  (reduce1);
    \draw[->,>=stealth',thick] (zData) to [out=-45,in=180] (reduce2);

    \draw[->,>=stealth',thick] (reduce1) to [out=0,in=180] (erg1);
    \draw[->,>=stealth',thick] (reduce2) to [out=0,in=180] (erg2);
  \end{tikzpicture}

  \end{center}
}

\subsection{Google Matrix}
\frame{
  \frametitle{Google Matrix / Page Rank}
\begin{center}
\only<1>{
  \begin{tikzpicture}
  \node[circle,fill=blue!20,draw=blue!50,ultra thick,inner sep=.5cm] {\Huge{Welt}};
\end{tikzpicture}}

\only<2>{
    %\column{.5\textwidth}
  \begin{tikzpicture}
  \node[circle,fill=blue!20,draw=blue!50,ultra thick,text width=1.5cm,align=center] (northAmerica) {Nord Amerika};
  \node[circle,fill=blue!20,draw=blue!50,ultra thick,text width=1.5cm,align=center] (southAmerica)[below=1mm of northAmerica] {Süd Amerika};
  \node[circle,fill=blue!20,draw=blue!50,ultra thick] (Europa) [right=1cm of northAmerica] {Europa};
  \node[circle,fill=blue!20,draw=blue!50,ultra thick] (Afrika)[below=5mm of Europa] {Afrika};
  \node[circle,fill=blue!20,draw=blue!50,ultra thick] (Asien) [right=1mm of Europa] {Asien};
  \node[circle,fill=blue!20,draw=blue!50,ultra thick] (Australien) [right=1cm of Afrika,yshift=-5mm] {Australien};
\end{tikzpicture}
}
\end{center}
}

\frame{
  \frametitle{Funktionierender Algorithmus}
\begin{center}
  \begin{tikzpicture}
  %\node[rectangle,fill=blue!20,draw=blue!50,ultra thick,align=center] (matrix) {pseudo Google Matrix generieren};
  %\node[rectangle,fill=red!20,draw=red!50,thick](TotMatrix)[below=1cm of matrix, xshift=-3cm]{};
  %\node[->,shape=arc,180:30:1cm,thick]()[below=1cm of matrix, xshift=-3cm]{all in one};
  \node[rectangle,fill=blue!20,draw=blue!50,ultra thick,align=center] (matrix) {pseudo Google Matrix generieren};
  \draw[->,>=stealth'](0,0) arc (0:30:3cm);
\end{tikzpicture}
\end{center}
}


%%% Local Variables: 
%%% mode: latex
%%% TeX-master: "main"
%%% End: 


\section{Eigenwert Berechnung}
%%% Local Variables: 
%%% mode: latex
%%% TeX-master: "00_main"
%%% End: 


\section{Eigenwert und Eigenvektor}

Was sind Eigenwert und Eigenvektor?  Folgendes Beispiel soll diese
Frage beantworten.

% Beispiel aus
% http://de.wikibooks.org/wiki/Mathematik:_Lineare_Algebra:_Eigenwerte_und_Eigenvektoren
\begin{beispiel}
  Gegeben seien:
  \begin{align*}
    A &= \begin{pmatrix}
           1 & 2\\
           0 & -2
         \end{pmatrix}
      &x &= \begin{pmatrix}
              -2\\
              3
            \end{pmatrix}
      &y &= \begin{pmatrix}
              1\\
              0
            \end{pmatrix} 
  \end{align*}
  Wir betrachten nun $A \cdot x$ und $A \cdot y$:
  \begin{align*}
    A \cdot x &= \begin{pmatrix}
                   1 & 2\\
                   0 & -2
                 \end{pmatrix}
                 \begin{pmatrix}
                   -2\\
                   3
                 \end{pmatrix}
               = \begin{pmatrix}
                   4\\
                   -6
                 \end{pmatrix}\\
    A \cdot y &= \begin{pmatrix}
                   1 & 2\\
                   0 & -2
                 \end{pmatrix}
                 \begin{pmatrix}
                   1\\
                   0
                 \end{pmatrix}
               = \begin{pmatrix}
                   1\\
                   0
                 \end{pmatrix}
  \end{align*}
  Interessant ist jetzt, dass:
  \begin{align*}
    A \cdot x &= -2x = (-2)\cdot\begin{pmatrix}
                                  -2\\
                                  3
                                \end{pmatrix}
                              = \begin{pmatrix}
                                  4\\
                                  -6
                                \end{pmatrix}\\
    A \cdot y &= 1y = 1\cdot\begin{pmatrix}
                              1\\
                              0
                            \end{pmatrix}
                          = \begin{pmatrix}
                              1\\
                              0
                            \end{pmatrix}
  \end{align*}
  Daraus ist ersichtlich, dass eine Matrix $A$, welche mit einem
  Vektor $x$ multipliziert wird, das gleiche Ergebnis erhält, wie wenn
  der Vektor $x$ mit einer Konstante multipliziert wird.  (Natürlich
  gilt das gleiche auch für $y$.)  Die Konstanten $1$ und $-2$ nennt
  man \begriff{Eigenwerte} der Matrix $A$.  Die Vektoren $x$ und $y$
  heissen \begriff{Eigenvektoren} der Matrix $A$.
\end{beispiel}

\section{Schluss}
\frame{
  \frametitle{Schlussbemerkung}

  \begin{itemize}
    \item Etwas parallelisieren ist aufwendig
    \item MapReduce für Teilparallelisierung grosser Daten
  \end{itemize}
}

\subsection{Fragen}
\frame{
  \frametitle{Fragen}
}




%%% Local Variables: 
%%% mode: latex
%%% TeX-master: "main"
%%% End: 


%%% END DOCUMENT
\end{document}