%%%%%%%%%%%%%%%%%%%%%%%%%%%%%%%%%%%%%%%%%%%%%%%%%%%%%%%%%%%%%%%%%%%%%%
%         FILE:  normalHeader.tex
%
%        USAGE:  ---
%
%  DESCRIPTION:  normal header for LaTeX files
%
%      OPTIONS:  ---
% REQUIREMENTS:  full LaTeX installations
%                main document
%                biber for bibliography
%         BUGS:  - change ngerman babel to nswissgerman (when
%                    cygwin/texlive is updated
%                - no support of auto language detection (with the
%                    current cygwin/texlive version)
%        NOTES:  ---
%       AUTHOR:  Pascal Stump, pascal-stump@bluewin.ch
%      VERSION:  see git repo
%      CREATED:  18.Apr.2014 - 17:50
%     REVISION:  see git repo
%%%%%%%%%%%%%%%%%%%%%%%%%%%%%%%%%%%%%%%%%%%%%%%%%%%%%%%%%%%%%%%%%%%%%%

%%% language definitions
\usepackage[T1]{fontenc}         % Umlaute as one character
\usepackage[utf8]{inputenc}      % character encoding

\usepackage[ngerman,      % main language in document
            english,             % other languages
           ngerman]
           {babel}               % correct language support
                                 %   change language within document :
                                 %     \selectlanguage{}

\usepackage[babel,               % use babel in background
            german=swiss]        % define german's style, also:
                                 %   quotes; guillemets; swiss
           {csquotes}            % quotes standardisation for whole
                                 %   document, usage: \enquote{}
                                 %   csquotes does not now
                                 %   nswissgerman, therefor definition:
\DeclareQuoteAlias{german}{nswissgerman}
                                 %   definition below only works with
                                 %     babel=german
\defineshorthand{"`}{\openautoquote}
\defineshorthand{"'}{\closeautoquote}

%%% miscellaneous
\usepackage{hyperref}            % url, clickable pdf

\usepackage[style=ieee,
            backend=biber,
            language=auto]
            {biblatex}           % citing of references
                                 %   usage:
                                 %     \bibliography{pathTo}
                                 %     \cite[prenote][postnote]{key}
                                 %     \printbibliography
\DeclareLanguageMapping{nswissgerman}{ngerman}

%%% preview
\usepackage[showlables,sections,floats,textmath,displaymath]{preview}

%%% picture & graphics
\usepackage{tikz}                % TikZ graphics
% \usetikzlibrary{arrows,        % include into main only what needed
%                 decorations,
%                 pathmorphing,
%                 backgrounds,
%                 positioning,
%                 fit,
%                 petri,
%                 calc,
%                 intersections,
%                 through}

%%% units
\usepackage{siunitx}             % SI unit  \si{unit} or \SI{value}{unit}
\sisetup{per-mode=symbol-or-fraction,
         detect-shape,
         range-phrase = { \translate{bis} } }
                                 %   fraction; font shape auto detect

%%% chemistry
\usepackage[version=3]{mhchem}   % easy typesetting of chemical
                                 %   formula, usage:
                                 %     \ce{1/2H20}
                                 %     \ce{^{227}_{90}Th+}

\usepackage{chemfig}             % drawing molecules (with TikZ)
                                 %   usage:
                                 %     \chemfig{}
% http://mirror.switch.ch/ftp/mirror/tex/macros/generic/chemfig/chemfig_doc_en.pdf

%%% electronic                   % drawing electrical circuits (with
\usepackage[europeanvoltages,
            europeancurrents,
            europeanresistors,   % rectangular shape
            americaninductors,   % "4-bumbs" shape
            europeanports,       % rectangular logic ports
            siunitx,             % #1<#2>
            emptydiodes,
            noarrowmos,
            smartlabels]         % lables are rotated in a smart way
           {circuitikz}          %   TikZ), usage:
                                 %   \begin{circuitikz}
% http://mirror.switch.ch/ftp/mirror/tex/graphics/pgf/contrib/circuitikz/circuitikzmanual.pdf

